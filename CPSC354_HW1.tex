**Level 5**
**Lean Proof**: [
    \documentclass{article}
    \usepackage{amsmath}
    \usepackage{amssymb}
    \usepackage{listings}  
    
    \lstset{  
      language=Lean,
      basicstyle=\ttfamily,
      keywordstyle=\color{blue},
      commentstyle=\color{gray},
      stringstyle=\color{green},
    }
    
    \begin{document}
    
    \section*{Proof in Lean}
    
    We want to prove the following statement in Lean:
    
    \[
    a + (b + 0) + (c + 0) = a + b + c
    \]
    
    Here is the proof in Lean code:
    
    \begin{lstlisting}
    example (a b c : ℕ) : a + (b + 0) + (c + 0) = a + b + c :=
    begin
      rw add_zero,
      rw add_zero,
      refl,
    end
    \end{lstlisting}
    
    \end{document}
    

]

**level 5 explanation**: [ 'rw [add_zero]' simplifies the expression b + 0 to b, based on the theorem that ading zero to any number returns that number. the second application simplifies c + 0 in the same manner.
by appplying these epxressions a + (b + 0) + (c + 0) is reduced to a + b + c, which defines that proof.]

**Level 6**
*Lean Proof**: [
\documentclass{article}
\usepackage{amsmath}
\usepackage{amssymb}
\usepackage{listings}  

\lstset{  
  language=Lean,
  basicstyle=\ttfamily,
  keywordstyle=\color{blue},
  commentstyle=\color{gray},
  stringstyle=\color{green},
}

\begin{document}

\section*{Proof in Lean with Precision Rewriting}


\[
a + (b + 0) + (c + 0) = a + b + c
\]


\begin{lstlisting}
example (a b c : ℕ) : a + (b + 0) + (c + 0) = a + b + c :=
begin
  rw add_zero,
  rw add_zero c,
  refl,
end
\end{lstlisting}

\end{document}

]

**Level 7**
*Lean Proof**: [
\documentclass{article}
\usepackage{amsmath}
\usepackage{amssymb}
\usepackage{listings}

\lstset{
  language=Lean,
  basicstyle=\ttfamily,
  keywordstyle=\color{blue},
  stringstyle=\color{green},
  commentstyle=\color{gray},
  breaklines=true
}

\begin{document}

\section*{Lean Proof: succ\_eq\_add\_one}

We aim to prove that for all natural numbers \(n\), the successor of \(n\) is equal to \(n + 1\):

\[
\text{succ } n = n + 1
\]


\begin{lstlisting}
theorem succ_eq_add_one (n : ℕ) : succ n = n + 1 :=
begin
  rw add_succ,
  refl,
end
\end{lstlisting}

\end{document}

]

**Level 8**
*Lean Proof**: [
\documentclass{article}
\usepackage{amsmath}
\usepackage{amssymb}
\usepackage{listings}
\usepackage{xcolor}

\lstset{
  language=Lean,
  basicstyle=\ttfamily,
  keywordstyle=\color{blue},
  stringstyle=\color{green},
  commentstyle=\color{gray},
  breaklines=true
}

\begin{document}

\section*{Lean Proof: \(2 + 2 = 4\)}

prove that \(2 + 2 = 4\) in natural numbers:

\[
2 + 2 = 4
\]


\begin{lstlisting}
example : (2 : ℕ) + 2 = 4 :=
begin
  rw [two_eq_succ_one],
  nth_rewrite 2 [two_eq_succ_one],
  rw [succ_add],
  refl,
end
\end{lstlisting}

\end{document}

]






